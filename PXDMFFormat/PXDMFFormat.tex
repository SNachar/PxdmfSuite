\documentclass[a4paper,11pt]{article}

\usepackage[latin1]{inputenc}
\usepackage{fancyvrb}
\usepackage{fullpage}
\usepackage[usenames]{color}

\newenvironment{itemize*}%
  {\begin{itemize}%
    \setlength{\itemsep}{0pt}%
    \setlength{\parskip}{0pt}}%
  {\end{itemize}}


\title{PXDMF : A File Format for Separated Variables Problems\\Version 1.6}
\author{Felipe Bordeu Weldt\\
\small Materials, Processing and Composites Technology Group\\[-0.8ex]
\small GeM Institute, Ecole Centrale de Nantes\\
\small \texttt{felipe.bordeu@ec-nantes.fr}\\
}

\date{11/10/2013}

\begin{document}

\maketitle
\section{Introduction}
This file format is based on the popular XDMF file format (www.xdmf.org).
The "P" in front come from separated representation methods (PGD,POD).  
PXDMF files are fully compliant of XDMF, but XDMF file are not compliant of PXDMF.
So a PXDMF file can be open with a XDMF file reader without any problem.

The motivation is the need for a standardized method to exchange scientific data of Separated Variables Problems between High Performance Computing codes. 
XDMF already propose a solution for exchanging scientific data of non Separated Problems.
This document is based from the documentation of the XDMF format.


PXDMF (as XDMF) categorizes data by two main attribute; size and function.
Data can be \emph{Light} (typically less than about thousand values) or \emph{ Heavy} (megabytes, terabytes, etc.).
In addition to raw values, data can refer to \emph{Format} (rank and dimensions of an array) or \emph{Model} (how that data is to be used .i.e. XYZ coordinates vs. Vector components).

PXDMF (as XDMF) uses XML to store Light data and to describe the data Model.
Heavy data can be stored in the XML file, in a HDF5 file, or in a plain binary file.
The data Format is stored redundantly in both XML and HDF5.
This Allows tools to parse XML to determine the resources that will be required to access the Heavy data.

At www.xdmf.org a C++ API is provided to read and write XDMF data.
This API is not necessary in order to produce or consume PXDMF data.
Currently several HPC codes that already produce HDF5 data, use native text output to produce the XML necessary for valid XMDF.
This routines can be easily modified to write and/or read PXDMF files.

XML The eXtensible Markup Language (XML) format is widely used for many purposes and is well documented at many sites.
There are numerous open source parsers available for XML.

In PXDMF (as XDMF) is case sensitive and is made of three major components : elements, entities, and processing information. 
In PXDMF we are primarily concerned with the elements.
Theses elements follow the basic form :

\begin{Verbatim}[frame=single]
<ElementTag
   AttributeName="AttributeValue"
   AttributeName="AttributeValue"
   ... >
   CData
</ElementTag>
\end{Verbatim}

Each element begins with a \verb+<tag>+ and ends with a \verb+</tag>+. Optionally there can be several "Name=Value" pairs which convey additional information. 
Between the \verb+<tag>+ and the \verb+</tag>+ there can be other \verb+<tag></tag>+ pairs and/or character data (CData).
CData is typically where the values are stored; like the actual text in an HTML document. 

 
Comments in XML start with a "\verb+<!--+" and end with a "\verb+-->+".

So \verb+<!--This is a Comment -->+. 


XML is said to be "well formed" if it is syntactically correct.
That means all of the quotes match, all elements have end element, etc.
XML is said to be "valid" if it conforms to the \emph{Schema} or \emph{DTD} defined at the head of the document.
For example, the schema might specify that elements type A can contain element B but not element C.
Verifying that the provided XML is well formed and/or valid are functions typically performed by the XML parser.
Additionally PXDMF (like XDMF) takes advantage of two extensions to XML:
\newline

\begin{itemize}
\item \textbf{Xinclude}

As opposed to entity references in XML (which is a basic substitution mechanism).
XInclude allows for the inclusion of files that are not well formed XML.
This means that with XInclude the included file could be well formed XML or perhaps a flat text file of values.
The syntax looks like this :

\begin{Verbatim}[frame=single]
<Xdmf Version="2.0" xmlns:xi="[http://www.w3.org/2001/XInclude]">
<xi:include href="Example3.xmf"/>
</Xdmf>
\end{Verbatim}

the xmlns:xi establishes a namespace xi.
Then anywhere within the Xdmf element, xi:include will pull in the URL.
\newline

\item  \textbf{XPath} 

This allows for elements in the XML document to reference specific element in a document.
For example:

The first Grid in the first Domain 

\begin{Verbatim}[frame=single]
/Xdmf/Domain/Grid
\end{Verbatim}

The tenth Grid ... (XPath is one based). 

\begin{Verbatim}[frame=single]
/Xdmf/Domain/Grid[10]
\end{Verbatim}

The first grid with an attribute Name which has a value of "PGD1"

\begin{Verbatim}[frame=single]
/Xdmf/Domain/Grid[@Name="PGD1"]
\end{Verbatim}

\end{itemize}

All valid PXDMF must appear between the \verb+<Xdmf>+ and the \verb+</Xdmf>+ tags. So a minimal (empty) PXDMF XML file would be : 

\begin{Verbatim}[frame=single]
<?xml version="1.0" ?>
<!DOCTYPE Xdmf SYSTEM "Xdmf.dtd" []>
<Xdmf Version="2.0">
</Xdmf>
\end{Verbatim}


While there exists an Pxdmf DTD and a Schema they are only necessary for validating parsers. For performance reasons, validation is typically disabled. 

\section{\textbf{PXDMF Elements}}

The organization of PXDMF begins with the Xdmf element (Because PXDMF is base in Xdmf and we want to keep compatibility).
So that parsers can distinguish from previous versions of XDMF, there exists a Version attribute (currently at 2.0).
Any element in XDMF can have a Name attribute or have a Reference attribute.
The Name attribute becomes important for grids while the Reference attribute is used to take advantage of the XPath facility (more detail on this later).
Xdmf elements contain one or more Domain elements (computational domain).
There is seldom motivation to have more than one Domain.
Until now all the application using the PXDMF format use only one Domain element.


A Domain can have one or more Grid elements.
Each Grid element correspond to a dimension in the separated representation.
Each Grid contains two or more Information elements, a Topology, Geometry, and zero or more Attribute elements.
The Information elements specifies the dimensionality of the space and the names of each dimension.
Topology specifies the connectivity of the grid while Geometry specifies the location of the grid nodes. 
Attribute elements are used to specify values such as scalars and vectors that are located at the node or cell center.


To specify actual values for connectivity, geometry, or attributes, XDMF defines a DataItem element. A DataItem can provide the actual values or provide the physical storage (which is typically an HDF5 file). 

\subsection{\textbf{DataItem}}

Even if Xdmf supports six different types of DataItems, PXDMF only use uniform DataItems.

Uniform is the simplest type to specifies a single array.
As with all XDMF elements, there are reasonable defaults wherever possible.
So the simplest DataItem would be :

\begin{Verbatim}[frame=single]
<DataItem Dimensions="3">
    1.0 2.0 3.0
</DataItem>
\end{Verbatim}

Since no \emph{ItemType} has been specified, Uniform has been assumed.
The default \emph{Format} is XML and the default \emph{NumberType} is a 32 but floating point value.
So the fully qualified DataItem for the same data would be:

\begin{Verbatim}[frame=single]
<DataItem ItemType="Uniform"
  Format="XML"
  NumberType="Float" Precision="4"
  Rank="1" Dimensions="3">
  1.0 2.0 3.0
</DataItem>
\end{Verbatim}


Since it is only practical to store a small amount of data values in the XML (ASCII), production codes typically write their data to HDF5 or binary file and specify the location in XML.
HDF5 is a hierarchical, self describing data format.
So an application can open an HDF5 file without any prior knowledge of the data and determine the dimensions and number type of all the arrays stored in the file.
PXDMF requires that this information also be stored redundantly in the XML so that applications need not have access to the actual heavy data in order to determine storage requirements. 
Binary is just a plain binary file without format.


For example, suppose an application stored a three dimensional array of pressure values at each iteration into an HDF5 file. The XML might be : 

\begin{Verbatim}[frame=single]
<DataItem ItemType="Uniform"
  Format="HDF"
  NumberType="Float" Precision="8"
  Dimensions="64 128 256">
  OutputData.h5:/Results/Iteration 100/Part 2/Pressure
</DataItem>
\end{Verbatim}

Dimensions are specified with the slowest varying dimension first (i.e. KJI order).
The HDF filename can be fully qualified, if it is not it is assumed to be located in the current directory or the same directory as the XML file. 

And this is the version of a DataItem for a binary file:

\begin{Verbatim}[frame=single]
<DataItem ItemType="Uniform"
  Format="Binary"
  Endian="Big" 
  Seek="48" 
  NumberType="Float" Precision="8"
  Dimensions="64 128 256"
  >
OutputData.bin
</DataItem>
\end{Verbatim}

The \Verb+Endian+ keyword is use to set the endiannes of the binary file (Little or Big) and the \Verb+Seek+ keyword is used to tell the starting point of the data (unit is byte).
So more than one DataItem can be stored in only one binary file.


\subsubsection{\textbf{XPATH}}

In the following examples we take advantage of the XPath facility to reference DataItem elements that have been previously specified : 

\begin{Verbatim}[frame=single]
<Topology TopologyType="Triangle" NumberOfElements="648"  >
<DataItem Reference="/Xdmf/Domain/Grid[1]/Topology/DataItem" />
</Topology>
\end{Verbatim}

This is very useful when have been preciously defined. 

\subsection{\textbf{Grid}}

The DataItem element is used to define the data format portion of PXDMF.
The data model portion of PXDMF begins with the Grid element.
A Grid is a container for information related 1D, 2D or 3D points, structured or unstructured connectivity, and assigned values.

Even if  the Grid of Xdmf supports 4 different types of GridType, PXDMF only use uniform GridTypes.

Uniform Grid elements are the simplest type and must contain  two or more \emph{Information} elements and a  \emph{Topology}  and a  \emph{Geometry} element.


A basic Grid element would be:

\begin{Verbatim}[frame=single]
<Grid Name="PGD1" >
<Information Name="Dims" Value="1" />
<Information Name="Dim0" Value="x" />
<Information Name="Unit0" Value="m" />
<Topology ...
<Geometry ...
</Grid>
\end{Verbatim}

The value of the Name attribute of the Grid element ( \verb+"PGD1"+) correspond to the name and number of the PGD dimension (starting form 1). 

The first Information elements  gives the dimensionality of the Grid (PGD dimension) 1, 2 or 3, for 1D, 2D or 3D spaces.
The others Information elements give the name of each dimension. 

And extra information flag was added to store de unit used for the dimension (this is optional).
In the example the coordinate X is expressed in meters. 

\subsection{\textbf{Topology}}

The Topology element is the same as the XDMF.
The Topology element describes the general organization of the data.
This is the part of the computational grid that is invariant with rotation, translation, and scale.
For structured grids, the connectivity is implicit.
For unstructured grids, if the connectivity differs from the standard, an Order may be specified.
Currently, the following Topology cell types are defined :

\begin{itemize}

\item \textbf{Linear}
\begin{itemize*}
\item Polyvertex - a group of unconnected points
\item Polyline - a group of line segments
\item Polygon
\item Triangle
\item Quadrilateral
\item Tetrahedron
\item Pyramid
\item Wedge
\item Hexahedron 
\end{itemize*}

\item\textbf{Quadratic}
\begin{itemize*}
\item Edge\_3 - Quadratic line with 3 nodes
\item Tri\_6
\item Quad\_8
\item Tet\_10
\item Pyramid\_13
\item Wedge\_15
\item Hex\_20 
\end{itemize*}
\item\textbf{Arbitrary}
\begin{itemize*}
\item Mixed - a mixture of unstructured cells 

\end{itemize*}

\item  \textbf{Structured}
\begin{itemize*}
\item 2DSMesh - Curvilinear
\item 2DRectMesh - Axis are perpendicular
\item 2DCoRectMesh - Axis are perpendicular and spacing is constant
\item 3DSMesh
\item 3DRectMesh
\item 3DCoRectMesh 
\end{itemize*}
\end{itemize}
There is a NodesPerElement attribute for the cell types where it is not implicit. For example, to define a group of Octagons, set TopologyType="Polygon" and NodesPerElement="8".
This type of elements is not very common, but Polylines with only 2 nodes correspond to 1D elements. 
For structured grid topologies, the connectivity is implicit. 
For unstructured topologies the Topology element must contain a DataItem that defines the connectivity :
\begin{Verbatim}[frame=single]
<Topology TopologyType="Quadrilateral" NumberOfElements="2" >
  <DataItem Format="XML" DataType="Int" Dimensions="2 4">
    0 1 2 3
    1 6 7 2
  </DataItem>
</Topology>
\end{Verbatim}
The connectivity defines the indexes into the XYZ geometry that define the cell. In this example, the two quads share an edge defined by the line from node 1 to node 2. 
A Topology element can define Dimensions or NumberOfElements; this is just added for clarity.

Mixed topologies must define the cell type of every element. 
If the cell type does not have an implicit number of nodes, it must be specified. 
In this example, we define a topology of three cells consisting of a Tet (cell type 6) a Polygon (cell type 3) and a Hex (cell type 9) :
\begin{Verbatim}[frame=single]
<Topology TopologyType="Mixed" NumberOfElements="3" >
  <DataItem Format="XML" DataType="Int" Dimensions="20">
    6     0 1 2 7
    3  4  4 5 6 7
    9     8 9 10 11 12 13 14 15
  </DataItem>
</Topology> 
\end{Verbatim}
Notice that the Polygon must define the number of nodes (4) before its connectivity. The cell type numbers are defined in the API documentation. 

\begin{tabular}{p{2em}llll}
Cell type & Name & Description & XDMF internal name\\
\hline
 1 	& POLYVERTEX 	& A Group of Points (Atoms)								 & XDMF\_POLYVERTEX\\
 2  & POLYLINE 		& Line Segments (Bonds)									 & XDMF\_POLYLINE\\
 3  & POLYGON  		& N Sided												 & XDMF\_POLYGON\\
 4  & TRIANGLE 		& 3 Edge Polygon 										 & XDMF\_TRI\\
 5  & QUADRILATERAL	& 4 Edge Polygon 										 & XDMF\_QUAD\\
 6  & TETRAHEDRON 	& 4 Triangular Faces 									 & XDMF\_TET\\
 7  & PYRAMID  		& 4 Triangles, 1 quad base 					 & XDMF\_PYRAMID\\
 8  & WEDGE  		& 2 Triangles base, 3 quads faces					 & XDMF\_WEDGE\\
 9  & HEXAHEDRON  	& 6 quadrilateral faces									 & XDMF\_HEX\\
 34 & EDGE\_3  			& 3 Node H.O.Line 							 & XDMF\_EDGE\_3\\
 36 & TRIANGLE\_6  		& 6 Node H.O Triangle 						 & XDMF\_TRI\_6\\
 37 & QUADRILATERAL\_8 	& 8 Node H.O Quadrilateral					 & XDMF\_QUAD\_8\\
 35 & QUADRILATERAL\_9 	& 8 Node H.O Quadrilateral                    & XDMF\_QUAD\_9\\
 38 & TETRAHEDRON\_10  	& 10 Node H.O. Tetrahedron 					 & XDMF\_TET\_10\\
 39 & PYRAMID\_13  		& 13 Node H.O. Pyramid 						 & XDMF\_PYRAMID\_13\\
 40 & WEDGE\_15  		& 15 Node H.O. Wedge  		 & XDMF\_WEDGE\_15\\
 41 & WEDGE\_18  		& 18 Node H.O. Wedge  			& XDMF\_WEDGE\_18\\
 48 & HEXAHEDRON\_20  	& 20 Node H.O. Hexahedron  	& XDMF\_HEX\_20\\
 49 & HEXAHEDRON\_24  	& 24 Node H.O. Hexahedron 	& XDMF\_HEX\_24\\
 50 & HEXAHEDRON\_27  	& 27 Node H.O. Hexahedron 	& XDMF\_HEX\_27
\end{tabular}

\subsection{\textbf{Geometry}}

The Geometry element describes the XYZ values of the mesh. The important attribute here is the organization of the points. The default is XYZ; an X,Y, and Z for each point starting at parametric index 0. Possible organizations are :
\begin{itemize*}
\item XYZ - Interlaced locations
\item XY - Z is set to 0.0
\item X\_Y\_Z - X,Y, and Z are separate arrays
\item VXVYVZ - Three arrays, one for each axis
\item ORIGIN\_DXDYDZ - Six Values : Ox,Oy,Oz + Dx,Dy,Dz 
\end{itemize*}

The following Geometry element defines 8 points :
\begin{Verbatim}[frame=single]
<Geometry GeometryType="XYZ">
  <DataItem Format="XML" Dimensions="2 4 3">
    0.0    0.0    0.0
    1.0    0.0    0.0
    1.0    1.0    0.0
    0.0    1.0    0.0 
    0.0    0.0    2.0
    1.0    0.0    2.0
    1.0    1.0    2.0
    0.0    1.0    2.0
  </DataItem>
</Geometry>
\end{Verbatim}

In the case that the Grid has a dimensionality of 1 the second and the third columns are ignored.
 
Together with the Grid and Topology element we now have enough to define a full PXDMF XML file that defines two quadrilaterals that share an edge (notice not all points are used), and only one PGD dimension:


\begin{Verbatim}[frame=single]
<?xml version="1.0" ?>
<!DOCTYPE Xdmf SYSTEM "Xdmf.dtd" []>
<Xdmf Version="2.0" xmlns:xi="[http://www.w3.org/2001/XInclude]">
<Domain>
<Grid Name="PGD1">
<Information Name="Dims" Value="3">
<Information Name="Dim0" Value="X">
<Information Name="Dim1" Value="Y">
<Information Name="Dim2" Value="Z">
<Topology TopologyType="Quadrilateral" NumberOfElements="2" >
<DataItem Format="XML" DataType="Int" Dimensions="2 4">
0 1 2 3
1 6 7 2
</DataItem>
</Topology>
<Geometry GeometryType="XYZ">
<DataItem Format="XML" Dimensions="2 4 3">
0.0    0.0    0.0
1.0    0.0    0.0
1.0    1.0    0.0
0.0    1.0    0.0
0.0    0.0    2.0
1.0    0.0    2.0
1.0    1.0    2.0
0.0    1.0    2.0
</DataItem>
</Geometry>
</Grid>
</Domain>
</Xdmf>
\end{Verbatim}

It is valid to have DataItem elements to be direct children of the Xdmf or Domain elements. This could be useful if several Grids share the same Geometry but have separate Topology :
\begin{Verbatim}[frame=single]
<?xml version="1.0" ?>
<!DOCTYPE Xdmf SYSTEM "Xdmf.dtd" []>

<Xdmf Version="2.0" xmlns:xi="[http://www.w3.org/2001/XInclude]">
<Domain>
<DataItem Name="Point Data" Format="XML" Dimensions="2 4 3">
0.0    0.0    0.0
1.0    0.0    0.0
1.0    1.0    0.0
0.0    1.0    0.0
0.0    0.0    2.0
1.0    0.0    2.0
1.0    1.0    2.0
0.0    1.0    2.0
</DataItem>
<Grid Name="PGD1">
<Information Name="Dims" Value="3">
<Information Name="Dim0" Value="X">
<Information Name="Dim1" Value="Y">
<Information Name="Dim2" Value="Z">
<Topology Type="Quadrilateral" NumberOfElements="2" >
<DataItem Format="XML" 
DataType="Int"
Dimensions="2 4">
0 1 2 3
1 6 7 2
</DataItem>
</Topology>
<Geometry Type="XYZ">
<DataItem Reference="XML">
/Xdmf/Domain/DataItem[@Name="Point Data"]
</DataItem>
</Geometry>
</Grid>
</Domain>
</Xdmf>
\end{Verbatim}


\subsection{\textbf{Attribute}}

The Attribute element defines values associated with the mesh. Currently supported types of values are :
\begin{itemize*}
\item Scalar
\item Vector
\item Tensor - 9 values expected
\item Tensor6 - a symmetrical tensor
\end{itemize*}
These values can be centered on :
\begin{itemize*}
\item  Node
\item  Cell
\end{itemize*}

Typically Attributes are assigned on the Node :

\begin{Verbatim}[frame=single]
<Attribute Name="Node Values_0" Center="Node">
<DataItem Format="XML" Dimensions="6 4">
100 200 300 400
500 600 600 700
800 900 1000 1100
1200 1300 1400 1500
1600 1700 1800 1900
2000 2100 2200 2300
</DataItem>
</Attribute>
\end{Verbatim}
Or assigned to the cell centers :

\begin{Verbatim}[frame=single]
<Attribute Name="Cell Values_0" Center="Cell">
<DataItem Format="XML" Dimensions="3">
 3000 2000 1000
 </DataItem>
</Attribute>
\end{Verbatim}

The number after the name of the attribute ( \verb+... Name="Cell Values_0"  ..+) correspond to the mode number in the separated representation.


\subsection{\textbf{Time}}

The \textbf{Time} element is valid for Xdmf file but not for PXDMF file.

\subsection{\textbf{Information}}

The Information elements are used to store information about the dimensionality of the grid and names/unit of each axis inside it.
But also can be used to store code or system specific information that needs to be stored with the data that does not map to the current data model.
This is intended for application specific information that can be ignore.
A good example might be the bounds of a grid for a use in visualization .
Information elements have a Name and Value Attribute.
If Value is non-existent the value is in the CDATA of the element:

\begin{Verbatim}[frame=single]
<Information Name="XBounds" Value="0.0 10.0"/>
<Information Name="Bounds"> 0.0 10.0 100.0 110.0 200.0 210.0 </Information>
\end{Verbatim}

Several items can be addressed using the Information element like time, units, descriptions, etc. without polluting the XDMF schema. If some of these get used extensively they may be promoted to XDMF elements in the future. 

\section{XML Element and Default(*) XML Attributes :}

\begin{itemize}
\item Attribute 
\begin{Verbatim}[frame=single]
 Name            (no default, composed by : 
                  name + "_" + modenumber i.e. "Depx_0") 
 AttributeType   *Scalar | Vector | Tensor | Tensor6 
 Center          *Node | Cell 
 \end{Verbatim}


\item  DataItem
\begin{Verbatim}[frame=single]
 Name            (no default)
 ItemType        *Uniform 
 Dimensions      (no default) in KJI Order
 NumberType      *Float | Int | UInt | Char | UChar
 Precision       1 | *4 | 8
 Format          *XML | HDF | Binary
 Endian          *Big | Little
 Seek            *0 | #
 Reference       (no default, used by the XPATH)
 \end{Verbatim}

\item  Domain   (Only one)
\begin{Verbatim}[frame=single]
 Name            (no default)
\end{Verbatim}


\item  Geometry 
\begin{Verbatim}[frame=single]
 GeometryType     *XYZ | XY | X_Y_Z | VxVyVz | Origin_DxDyDz
\end{Verbatim}


\item Grid 
\begin{Verbatim}[frame=single]
 Name             (no default) composed by : "PGD" + thedimensionnumber
 		  (starting at 1 i.e. "PGD1")
 GridType         *Uniform 
\end{Verbatim}

\item  Information 
\begin{Verbatim}[frame=single]
 Name              (no default)
 Value             (no default)
\end{Verbatim}


\item Xdmf (XdmfRoot) 
\begin{Verbatim}[frame=single]
 Version            *Current Version | x
\end{Verbatim}

\item Topology 
\begin{Verbatim}[frame=single]
 Name               (no default)
 TopologyType       Polyvertex | Polyline | Polygon |
                    Triangle | Quadrilateral | Tetrahedron | Pyramid |
		    Wedge | Hexahedron | Edge_3 | Triagle_6 |
		    Quadrilateral_8 | Tetrahedron_10 | Pyramid_13 |
                    Wedge_15 | Hexahedron_20 |
                    Mixed |
                    2DSMesh | 2DRectMesh | 2DCoRectMesh |
                    3DSMesh | 3DRectMesh | 3DCoRectMesh
 NodesPerElement    (no default) Only Important for Polyvertex, Polygon 
                    and Polyline
 NumberOfElement    (no default)
     OR
 Dimensions         (no default)
 Order              each cell type has its own default
 BaseOffset         *0 | #
\end{Verbatim}

\end{itemize}

\section{Example of a PXDMF File for a canonic representation}

This example correspond to a 4D problem $(x,y,z,w)$, the problem is separated in $(x,y) \times (z) \times (w)$.
The scalar field $dep_x$ has only one mode (so only one attribute named "dep\_x\_0").

Then the solution can be reconstructed using:
\begin{equation}
dep\_x(x,y,z,w) = \sum_{i=0}^0(dep\_x_i(x,y)\cdot dep\_x_i(z)\cdot dep\_x_i(w))
\end{equation}
 
This file is pure XML (ASCII).
 
\begin{Verbatim}[frame=single]
<?xml version="1.0" ?>
<!DOCTYPE Xdmf SYSTEM "Xdmf.dtd" []>
<Xdmf Version="2.0" xmlns:xi="http://www.w3.org/2001/XInclude" >
  <Domain Name="test_separe3.pxdmf">
    <Grid Name="PGD1"  >
      <Information Name="Dims" Value="2"  />
      <Information Name="Dim0" Value="X"  />
      <Information Name="Unit0" Value="m"  />
      <Information Name="Dim1" Value="Y"  />
      <Information Name="Unit1" Value="m"  />
      <Topology TopologyType="Quadrilateral" NumberOfElements="2" >
        <DataItem  Format="XML" NumberType="int" Dimensions="2 4">
0 1 4 3
1 2 5 4	 
        </DataItem>
      </Topology>
      <Geometry GeometryType="XYZ">
        <DataItem  Format="XML" NumberType="float" Dimensions="6 3">
0   0   0
1   0   0
2   0   0
0   1   0
1.2 0.8 0
2   1   0
        </DataItem>
      </Geometry>
      <Attribute Name="dep_x_0" Center="Node"  AttributeType="Scalar"  >
        <DataItem  Format="XML" NumberType="float" Dimensions="1 6">
0 1 2 0.1 1.1 2.1
        </DataItem>
      </Attribute>
    </Grid>
    <Grid Name="PGD2"  >
      <Information Name="Dims" Value="1"  />
      <Information Name="Dim0" Value="Z"  />
      <Information Name="Unit0" Value="m"  />
      <Topology TopologyType="Polyline" NumberOfElements="3" 
        NodesPerElement="2"  >
        <DataItem  Format="XML" NumberType="int" Dimensions="3 2">
0 1
1 2
2 3
        </DataItem>
      </Topology>
      <Geometry GeometryType="XYZ">
        <DataItem  Format="XML" NumberType="float" Dimensions="4 3">
0.00000000  0.00000000  0.00000000
0.00694444  0.00000000  0.00000000
0.01388889  0.00000000  0.00000000
0.02083333  0.00000000  0.00000000
        </DataItem>
      </Geometry>
      <Attribute Name="dep_x_0" Center="Node"  AttributeType="Scalar"  >
        <DataItem  Format="XML" NumberType="float" Dimensions="1 4">
0.0 1.0 1.2 1.22
        </DataItem>
        </Attribute>
    </Grid>
    <Grid Name="PGD3"  >
      <Information Name="Dims" Value="1"  />
      <Information Name="Dim0" Value="W"  />
	  <Information Name="Unit0" Value="N"  />
      <Topology TopologyType="Polyline" NumberOfElements="4" 
         NodesPerElement="2" >
        <DataItem  Format="XML" NumberType="int" Dimensions="4 2">
0 1
1 2
2 3
3 4
        </DataItem>
      </Topology>
      <Geometry GeometryType="XYZ">
        <DataItem  Format="XML" NumberType="float" Dimensions="5 3">
0.0 0.0 0.0
0.1 0.0 0.0
0.2 0.0 0.0
0.3 0.0 0.0
0.5 0.0 0.0 
        </DataItem>
      </Geometry>
      <Attribute Name="dep_x_0" Center="Node"  AttributeType="Scalar"  >
        <DataItem  Format="XML" NumberType="float" Dimensions="1 5">
0.0 1.0 1.2 1.22 1.222
        </DataItem>
      </Attribute>
    </Grid>
  </Domain>
</Xdmf>
\end{Verbatim}

%---------------------------------------------------------------------------------------------\\
%--------------------THIS PART IS WORK IN PROGRESS-------------------------------------------\\
%------------------------------------------------------------------------------------------------------------
%\section{Example of a PXDMF File for a Tucker representation}
%
%This example correspond to a 4D problem $(x,y,z,w)$, the problem is separated in $(x,y) \times (z) \times (w)$.
%The field "dep\_x" has only two modes (so two attributes named "dep\_x\_0" and "dep\_x\_1").
%
%Then the solution can be reconstructed using:
%\begin{equation}
%dep\_x(x,y,z,w) = \sum_{i=0}^1\sum_{j=0}^1\sum_{k=0}^1 \alpha_{ijk}(dep\_x_i(x,y)\cdot dep\_x_j(z)\cdot dep\_x_k(w))
%\end{equation}
% 
% 
%This file is pure XML (ASCII).
% 
%\begin{Verbatim}[frame=single]
%<?xml version="1.0" ?>
%<!DOCTYPE Xdmf SYSTEM "Xdmf.dtd" []>
%<Xdmf Version="2.0" xmlns:xi="http://www.w3.org/2001/XInclude" >
%  <Domain Name="test_separe3.pxdmf">
%    <Information Name="Type" Value="Tucker"  />
%    <DataItem Name="alpha_point_dep_x" Format="XML" NumberType="float" Dimensions="2 2 2">
%1.00000000  2.00000000  
%3.00000000  4.00694444  
%
%1.10000000  2.20000000
%3.31388889  4.40000000  
%
%1.11000000  2.22083333  
%3.33000000  4.44000000
%    </DataItem>
%    <Grid Name="PGD1"  >
%      <Information Name="Dims" Value="2"  />
%      <Information Name="Dim0" Value="X"  />
%      <Information Name="Unit0" Value="m"  />
%      <Information Name="Dim1" Value="Y"  />
%      <Information Name="Unit1" Value="m"  />
%      <Topology TopologyType="Quadrilateral" NumberOfElements="2" >
%        <DataItem  Format="XML" NumberType="int" Dimensions="2 4">
%0 1 4 3
%1 2 5 4	 
%        </DataItem>
%      </Topology>
%      <Geometry GeometryType="XYZ">
%        <DataItem  Format="XML" NumberType="float" Dimensions="6 3">
%0   0   0
%1   0   0
%2   0   0
%0   1   0
%1.2 0.8 0
%2   1   0
%        </DataItem>
%      </Geometry>
%      <Attribute Name="dep_x_0" Center="Node"  AttributeType="Scalar"  >
%        <DataItem  Format="XML" NumberType="float" Dimensions="1 6">
%0 1 2 0.1 1.1 2.1
%        </DataItem>
%      </Attribute>
%    </Grid>
%    <Grid Name="PGD2"  >
%      <Information Name="Dims" Value="1"  />
%      <Information Name="Dim0" Value="Z"  />
%      <Information Name="Unit0" Value="m"  />
%      <Topology TopologyType="Polyline" NumberOfElements="3" 
%        NodesPerElement="2"  >
%        <DataItem  Format="XML" NumberType="int" Dimensions="3 2">
%0 1
%1 2
%2 3
%        </DataItem>
%      </Topology>
%      <Geometry GeometryType="XYZ">
%        <DataItem  Format="XML" NumberType="float" Dimensions="4 3">
%0.00000000  0.00000000  0.00000000
%0.00694444  0.00000000  0.00000000
%0.01388889  0.00000000  0.00000000
%0.02083333  0.00000000  0.00000000
%        </DataItem>
%      </Geometry>
%      <Attribute Name="dep_x_0" Center="Node"  AttributeType="Scalar"  >
%        <DataItem  Format="XML" NumberType="float" Dimensions="1 4">
%0.0 1.0 1.2 1.22
%        </DataItem>
%        </Attribute>
%    </Grid>
%    <Grid Name="PGD3"  >
%      <Information Name="Dims" Value="1"  />
%      <Information Name="Dim0" Value="W"  />
%	  <Information Name="Unit0" Value="N"  />
%      <Topology TopologyType="Polyline" NumberOfElements="4" 
%         NodesPerElement="2" >
%        <DataItem  Format="XML" NumberType="int" Dimensions="4 2">
%0 1
%1 2
%2 3
%3 4
%        </DataItem>
%      </Topology>
%      <Geometry GeometryType="XYZ">
%        <DataItem  Format="XML" NumberType="float" Dimensions="5 3">
%0.0 0.0 0.0
%0.1 0.0 0.0
%0.2 0.0 0.0
%0.3 0.0 0.0
%0.5 0.0 0.0 
%        </DataItem>
%      </Geometry>
%      <Attribute Name="dep_x_0" Center="Node"  AttributeType="Scalar"  >
%        <DataItem  Format="XML" NumberType="float" Dimensions="1 5">
%0.0 1.0 1.2 1.22 1.222
%        </DataItem>
%      </Attribute>
%    </Grid>
%  </Domain>
%</Xdmf>
%\end{Verbatim}


\end{document}
